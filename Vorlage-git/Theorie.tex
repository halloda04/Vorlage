\section{Theoretische Grundlagen}

Bei dem Transport von Wärmeenergie gibt es nur 
drei unterschiedliche Phänomene. Das erste ist die 
Wärmestrahlung, hierbei wird Wärmeenergie über 
elektromagnetische Wellen, meist im Infrarotbereich, 
transportiert. Dies ist aber sehr inneffizient im Vergleich 
zu den anderen Arten des Wärmetransports. Aufgrund dessen 
ist es ziemlich komplex vernünftige Kühlsysteme innerhalb 
eines Vakuums zu entwickeln. Die zweite Art des 
Wärmetransports nennt sich Konvektion, diese beschreibt die 
Bewegung von Gasen und Flüssigkeiten, die sich aufgrund von 
Temperaturschwankungen innerhalb eines Raumes bewegen. Die 
dritte und letzte Art des Wärmetransports ist die 
Wärmeleitung. Die Wärmeleitung funktioniert Aufgrund von 
Stößen der im Material vorhandenen Atomkerne und Elektronen 
dieser Energietransport funktioniert sehr Effizient 
innerhalb von Metallen, da in diesen die Elektronen sehr 
beweglich sind und somit auch viele Stöße passieren können, 
oder Materialien bei den die Atomkerne sehr nah bei einander 
liegen.\\
In diesem Versuhc wird die Wärmeleitung näher Betrachtet, sie wird mathematisch durch folgende Gleichung ausgedrückt:\\

\begin{equation}
\frac{\partial T}{\partial t}=\frac{\lambda}{c\rho}\nabla^2 T
\end{equation}
In dieser Gleichung ist T die Temperatur, $\delta$ die 
Wärmleitfähigkeitskontante, c die Spezifische 
Wärmekapazität des Materials und $\rho$ die Dichte des 
untersuchten Materials. Im einfacheren ein-dimensionalen fall gibt es eine kleine abwandlung der Formel\\
\begin{equation}
\frac{\partial T}{\partial t}=\frac{\lambda}{c\rho}\frac{\partial^2 T}{\partial x^2}
\end{equation}

