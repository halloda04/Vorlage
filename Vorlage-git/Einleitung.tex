\section{Einleitung}

Die Wärmeleitung ist ein grundlegendes Prinzip der Physik 
und sie kommt in allen möglichen Bereichen des Lebens vor. 
Sowohl im täglichen Leben als auch in der Wissenschaft. Im 
täglichen Leben tritt sie bei so simplen Dingen wie heißem 
Tee oder auch dem Kochen auf, und in der Wissenschaft bei 
den Niedrigtemperatur-Mikroskopen vom Experimental-Physik 5 
und bei den Plasmaexperimenten der Plasma-Physik in unserem 
Institut. Wärme ist ein wirklich alltägliches Phänomen was 
bei eigentlich allem anfällt was irgendwie irgendwo 
passiert. Je genauer der Leitfähigkeit eines Stoffes 
bekannt ist desto besser kann also Technik die sich mit dem 
Transport von Wärme beschäftigt wie Kühlgeräte oder 
Klimanlagen weiterentwickelt werden. Der Versuch ist dem 
Gebiet der Wärmelehre unter zuordnen. Dementsprechen sind 
mögliche Messgrößen mit Thermomether und Stopuhr 
feststellbar. 